\documentclass[11pt, a4paper]{article}
%Title
\title{Infinitesimal Transitions Through Chaotic Systems: A Theoretical Exploration}
\author{Gavin Lemieux}
\date{}


\begin{document}
	%introductory pages
	\maketitle

	\section{Overview}
	This theoretical exploration delves into the concept of how infinitesimal steps can facilitate a smooth transition through chaotic systems. Inspired by visualizations from an intriguing YouTube video, the premise is that by examining a system at the infinitesimal level, we can trace how one path subtly evolves into another, despite the apparent chaos.

	\section{Core Concept}
	In chaotic systems, even the smallest differences in initial conditions can lead to vastly different outcomes. These systems often exhibit immediate divergences or bifurcations, where the evolution of their states becomes unpredictable. However, by zooming in on infinitesimally small intervals between these points, we might discover that the transition between them is surprisingly smooth, even though the overall system is chaotic.

	For instance, imagine a chaotic system where the initial conditions of two expressions are almost identical, differing only by a minute factor such as 0.00001. At first glance, these systems appear nearly identical, but as time progresses, they begin to diverge. Yet, if we examine the system at an infinitesimal scale, we can observe how the differences evolve gradually. By considering midpoints between the two paths, and analyzing the infinitesimal steps that lead to these points, we can trace how both paths slowly converge toward a common outcome.

	\section{Key Ideas}
	\subsection{Infinitesimal Transitions}
		At an infinitesimal scale, the differences between two paths can be so small that the systems appear nearly identical. It’s only when we expand the view to a larger scale, or map the functions toward some amonut of infinity dependent on the scale of the infinitesimal steps, that the divergence between the two paths becomes apparent.

	\subsection{Scaling and Divergence:} 
		As we zoom out, the smooth transition becomes less pronounced. The divergence between the systems increases over time, but with smaller and smaller infinitesimal steps, the shift remains gradual, preventing a sudden, jarring change. The smallest infinitesimal steps will keep the functions similar for a much larger infinite amount of time.

	\subsection{Potential for Further Research:} 
		This idea opens up interesting questions around chaotic dynamics, potential fractal-like behavior, and how we could quantify or visualize these transitions.

	\section{Possible Applications and Further Exploration}
	\subsection{Mapping Chaos}
	It may be useful in mapping chaotic trajectories that can be represented by smooth changes despite their unpredictable behavior at larger scales.

	\section{Conclusion}
	While chaotic systems appear random and unpredictable, the infinitesimal path between any two points may have hidden patterns about their true nature. This exploration into the smooth transitions within chaotic systems is a theoretical idea that could lead to a deeper understanding of how chaos evolves at the micro level, with potential applications in modeling complex systems.